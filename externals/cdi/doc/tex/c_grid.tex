

\subsection{Create a horizontal Grid: \texttt{gridCreate}}
\index{gridCreate}
\label{gridCreate}

The function {\texttt{gridCreate}} creates a horizontal Grid.

\subsubsection*{Usage}

\begin{verbatim}
    int gridCreate(int gridtype, SizeType size);
\end{verbatim}

\hspace*{4mm}\begin{minipage}[]{15cm}
\begin{deflist}{\texttt{gridtype}\ }
\item[\texttt{gridtype}]
The type of the grid, one of the set of predefined {\CDI} grid types.
                     The valid {\CDI} grid types are {\texttt{GRID\_GENERIC}}, {\texttt{GRID\_LONLAT}},
                     {\texttt{GRID\_GAUSSIAN}}, {\texttt{GRID\_PROJECTION}}, {\texttt{GRID\_SPECTRAL}},
                     {\texttt{GRID\_GME}}, {\texttt{GRID\_CURVILINEAR}} and {\texttt{GRID\_UNSTRUCTURED}}.
\item[\texttt{size}]
Number of gridpoints.

\end{deflist}
\end{minipage}

\subsubsection*{Result}

{\texttt{gridCreate}} returns an identifier to the Grid.


\subsubsection*{Example}

Here is an example using {\texttt{gridCreate}} to create a regular lon/lat Grid:

\begin{lstlisting}[language=C, backgroundcolor=\color{pyellow}, basicstyle=\small, columns=flexible]

    #include "cdi.h"
       ...
    #define  nlon  12
    #define  nlat   6
       ...
    double lons[nlon] = {0, 30, 60, 90, 120, 150, 180, 210, 240, 270, 300, 330};
    double lats[nlat] = {-75, -45, -15, 15, 45, 75};
    int gridID;
       ...
    gridID = gridCreate(GRID_LONLAT, nlon*nlat);
    gridDefXsize(gridID, nlon);
    gridDefYsize(gridID, nlat);
    gridDefXvals(gridID, lons);
    gridDefYvals(gridID, lats);
       ...
\end{lstlisting}


\subsection{Destroy a horizontal Grid: \texttt{gridDestroy}}
\index{gridDestroy}
\label{gridDestroy}
\subsubsection*{Usage}

\begin{verbatim}
    void gridDestroy(int gridID);
\end{verbatim}

\hspace*{4mm}\begin{minipage}[]{15cm}
\begin{deflist}{\texttt{gridID}\ }
\item[\texttt{gridID}]
Grid ID, from a previous call to {\htmlref{\texttt{gridCreate}}{gridCreate}}.

\end{deflist}
\end{minipage}


\subsection{Duplicate a horizontal Grid: \texttt{gridDuplicate}}
\index{gridDuplicate}
\label{gridDuplicate}

The function {\texttt{gridDuplicate}} duplicates a horizontal Grid.

\subsubsection*{Usage}

\begin{verbatim}
    int gridDuplicate(int gridID);
\end{verbatim}

\hspace*{4mm}\begin{minipage}[]{15cm}
\begin{deflist}{\texttt{gridID}\ }
\item[\texttt{gridID}]
Grid ID, from a previous call to {\htmlref{\texttt{gridCreate}}{gridCreate}} or {\htmlref{\texttt{vlistInqVarGrid}}{vlistInqVarGrid}}.

\end{deflist}
\end{minipage}

\subsubsection*{Result}

{\texttt{gridDuplicate}} returns an identifier to the duplicated Grid.



\subsection{Get the type of a Grid: \texttt{gridInqType}}
\index{gridInqType}
\label{gridInqType}

The function {\texttt{gridInqType}} returns the type of a Grid.

\subsubsection*{Usage}

\begin{verbatim}
    int gridInqType(int gridID);
\end{verbatim}

\hspace*{4mm}\begin{minipage}[]{15cm}
\begin{deflist}{\texttt{gridID}\ }
\item[\texttt{gridID}]
Grid ID, from a previous call to {\htmlref{\texttt{gridCreate}}{gridCreate}} or {\htmlref{\texttt{vlistInqVarGrid}}{vlistInqVarGrid}}.

\end{deflist}
\end{minipage}

\subsubsection*{Result}

{\texttt{gridInqType}} returns the type of the grid,
one of the set of predefined {\CDI} grid types.
The valid {\CDI} grid types are {\texttt{GRID\_GENERIC}}, {\texttt{GRID\_LONLAT}},
{\texttt{GRID\_GAUSSIAN}}, {\texttt{GRID\_PROJECTION}}, {\texttt{GRID\_SPECTRAL}}, {\texttt{GRID\_GME}},
{\texttt{GRID\_CURVILINEAR}} and {\texttt{GRID\_UNSTRUCTURED}}.



\subsection{Get the size of a Grid: \texttt{gridInqSize}}
\index{gridInqSize}
\label{gridInqSize}

The function {\texttt{gridInqSize}} returns the size of a Grid.

\subsubsection*{Usage}

\begin{verbatim}
    SizeType gridInqSize(int gridID);
\end{verbatim}

\hspace*{4mm}\begin{minipage}[]{15cm}
\begin{deflist}{\texttt{gridID}\ }
\item[\texttt{gridID}]
Grid ID, from a previous call to {\htmlref{\texttt{gridCreate}}{gridCreate}} or {\htmlref{\texttt{vlistInqVarGrid}}{vlistInqVarGrid}}.

\end{deflist}
\end{minipage}

\subsubsection*{Result}

{\texttt{gridInqSize}} returns the number of grid points of a Grid.



\subsection{Define the number of values of a X-axis: \texttt{gridDefXsize}}
\index{gridDefXsize}
\label{gridDefXsize}

The function {\texttt{gridDefXsize}} defines the number of values of a X-axis.

\subsubsection*{Usage}

\begin{verbatim}
    void gridDefXsize(int gridID, SizeType xsize);
\end{verbatim}

\hspace*{4mm}\begin{minipage}[]{15cm}
\begin{deflist}{\texttt{gridID}\ }
\item[\texttt{gridID}]
Grid ID, from a previous call to {\htmlref{\texttt{gridCreate}}{gridCreate}}.
\item[\texttt{xsize}]
Number of values of a X-axis.

\end{deflist}
\end{minipage}


\subsection{Get the number of values of a X-axis: \texttt{gridInqXsize}}
\index{gridInqXsize}
\label{gridInqXsize}

The function {\texttt{gridInqXsize}} returns the number of values of a X-axis.

\subsubsection*{Usage}

\begin{verbatim}
    SizeType gridInqXsize(int gridID);
\end{verbatim}

\hspace*{4mm}\begin{minipage}[]{15cm}
\begin{deflist}{\texttt{gridID}\ }
\item[\texttt{gridID}]
Grid ID, from a previous call to {\htmlref{\texttt{gridCreate}}{gridCreate}} or {\htmlref{\texttt{vlistInqVarGrid}}{vlistInqVarGrid}}.

\end{deflist}
\end{minipage}

\subsubsection*{Result}

{\texttt{gridInqXsize}} returns the number of values of a X-axis.



\subsection{Define the number of values of a Y-axis: \texttt{gridDefYsize}}
\index{gridDefYsize}
\label{gridDefYsize}

The function {\texttt{gridDefYsize}} defines the number of values of a Y-axis.

\subsubsection*{Usage}

\begin{verbatim}
    void gridDefYsize(int gridID, SizeType ysize);
\end{verbatim}

\hspace*{4mm}\begin{minipage}[]{15cm}
\begin{deflist}{\texttt{gridID}\ }
\item[\texttt{gridID}]
Grid ID, from a previous call to {\htmlref{\texttt{gridCreate}}{gridCreate}}.
\item[\texttt{ysize}]
Number of values of a Y-axis.

\end{deflist}
\end{minipage}


\subsection{Get the number of values of a Y-axis: \texttt{gridInqYsize}}
\index{gridInqYsize}
\label{gridInqYsize}

The function {\texttt{gridInqYsize}} returns the number of values of a Y-axis.

\subsubsection*{Usage}

\begin{verbatim}
    SizeType gridInqYsize(int gridID);
\end{verbatim}

\hspace*{4mm}\begin{minipage}[]{15cm}
\begin{deflist}{\texttt{gridID}\ }
\item[\texttt{gridID}]
Grid ID, from a previous call to {\htmlref{\texttt{gridCreate}}{gridCreate}} or {\htmlref{\texttt{vlistInqVarGrid}}{vlistInqVarGrid}}.

\end{deflist}
\end{minipage}

\subsubsection*{Result}

{\texttt{gridInqYsize}} returns the number of values of a Y-axis.



\subsection{Define the number of parallels between a pole and the equator: \texttt{gridDefNP}}
\index{gridDefNP}
\label{gridDefNP}

The function {\texttt{gridDefNP}} defines the number of parallels between a pole and the equator
of a Gaussian grid.

\subsubsection*{Usage}

\begin{verbatim}
    void gridDefNP(int gridID, int np);
\end{verbatim}

\hspace*{4mm}\begin{minipage}[]{15cm}
\begin{deflist}{\texttt{gridID}\ }
\item[\texttt{gridID}]
Grid ID, from a previous call to {\htmlref{\texttt{gridCreate}}{gridCreate}}.
\item[\texttt{np}]
Number of parallels between a pole and the equator.

\end{deflist}
\end{minipage}


\subsection{Get the number of parallels between a pole and the equator: \texttt{gridInqNP}}
\index{gridInqNP}
\label{gridInqNP}

The function {\texttt{gridInqNP}} returns the number of parallels between a pole and the equator
of a Gaussian grid.

\subsubsection*{Usage}

\begin{verbatim}
    int gridInqNP(int gridID);
\end{verbatim}

\hspace*{4mm}\begin{minipage}[]{15cm}
\begin{deflist}{\texttt{gridID}\ }
\item[\texttt{gridID}]
Grid ID, from a previous call to {\htmlref{\texttt{gridCreate}}{gridCreate}} or {\htmlref{\texttt{vlistInqVarGrid}}{vlistInqVarGrid}}.

\end{deflist}
\end{minipage}

\subsubsection*{Result}

{\texttt{gridInqNP}} returns the number of parallels between a pole and the equator.



\subsection{Define the values of a X-axis: \texttt{gridDefXvals}}
\index{gridDefXvals}
\label{gridDefXvals}

The function {\texttt{gridDefXvals}} defines all values of the X-axis.

\subsubsection*{Usage}

\begin{verbatim}
    void gridDefXvals(int gridID, const double *xvals);
\end{verbatim}

\hspace*{4mm}\begin{minipage}[]{15cm}
\begin{deflist}{\texttt{gridID}\ }
\item[\texttt{gridID}]
Grid ID, from a previous call to {\htmlref{\texttt{gridCreate}}{gridCreate}}.
\item[\texttt{xvals}]
X-values of the grid.

\end{deflist}
\end{minipage}


\subsection{Get all values of a X-axis: \texttt{gridInqXvals}}
\index{gridInqXvals}
\label{gridInqXvals}

The function {\texttt{gridInqXvals}} returns all values of the X-axis.

\subsubsection*{Usage}

\begin{verbatim}
    SizeType gridInqXvals(int gridID, double *xvals);
\end{verbatim}

\hspace*{4mm}\begin{minipage}[]{15cm}
\begin{deflist}{\texttt{gridID}\ }
\item[\texttt{gridID}]
Grid ID, from a previous call to {\htmlref{\texttt{gridCreate}}{gridCreate}} or {\htmlref{\texttt{vlistInqVarGrid}}{vlistInqVarGrid}}.
\item[\texttt{xvals}]
Pointer to the location into which the X-values are read.
                    The caller must allocate space for the returned values.

\end{deflist}
\end{minipage}

\subsubsection*{Result}

Upon successful completion {\texttt{gridInqXvals}} returns the number of values and
the values are stored in {\texttt{xvals}}.
Otherwise, 0 is returned and {\texttt{xvals}} is empty.



\subsection{Define the values of a Y-axis: \texttt{gridDefYvals}}
\index{gridDefYvals}
\label{gridDefYvals}

The function {\texttt{gridDefYvals}} defines all values of the Y-axis.

\subsubsection*{Usage}

\begin{verbatim}
    void gridDefYvals(int gridID, const double *yvals);
\end{verbatim}

\hspace*{4mm}\begin{minipage}[]{15cm}
\begin{deflist}{\texttt{gridID}\ }
\item[\texttt{gridID}]
Grid ID, from a previous call to {\htmlref{\texttt{gridCreate}}{gridCreate}}.
\item[\texttt{yvals}]
Y-values of the grid.

\end{deflist}
\end{minipage}


\subsection{Get all values of a Y-axis: \texttt{gridInqYvals}}
\index{gridInqYvals}
\label{gridInqYvals}

The function {\texttt{gridInqYvals}} returns all values of the Y-axis.

\subsubsection*{Usage}

\begin{verbatim}
    SizeType gridInqYvals(int gridID, double *yvals);
\end{verbatim}

\hspace*{4mm}\begin{minipage}[]{15cm}
\begin{deflist}{\texttt{gridID}\ }
\item[\texttt{gridID}]
Grid ID, from a previous call to {\htmlref{\texttt{gridCreate}}{gridCreate}} or {\htmlref{\texttt{vlistInqVarGrid}}{vlistInqVarGrid}}.
\item[\texttt{yvals}]
Pointer to the location into which the Y-values are read.
                    The caller must allocate space for the returned values.

\end{deflist}
\end{minipage}

\subsubsection*{Result}

Upon successful completion {\texttt{gridInqYvals}} returns the number of values and
the values are stored in {\texttt{yvals}}.
Otherwise, 0 is returned and {\texttt{yvals}} is empty.



\subsection{Define the bounds of a X-axis: \texttt{gridDefXbounds}}
\index{gridDefXbounds}
\label{gridDefXbounds}

The function {\texttt{gridDefXbounds}} defines all bounds of the X-axis.

\subsubsection*{Usage}

\begin{verbatim}
    void gridDefXbounds(int gridID, const double *xbounds);
\end{verbatim}

\hspace*{4mm}\begin{minipage}[]{15cm}
\begin{deflist}{\texttt{xbounds}\ }
\item[\texttt{gridID}]
Grid ID, from a previous call to {\htmlref{\texttt{gridCreate}}{gridCreate}}.
\item[\texttt{xbounds}]
X-bounds of the grid.

\end{deflist}
\end{minipage}


\subsection{Get the bounds of a X-axis: \texttt{gridInqXbounds}}
\index{gridInqXbounds}
\label{gridInqXbounds}

The function {\texttt{gridInqXbounds}} returns the bounds of the X-axis.

\subsubsection*{Usage}

\begin{verbatim}
    SizeType gridInqXbounds(int gridID, double *xbounds);
\end{verbatim}

\hspace*{4mm}\begin{minipage}[]{15cm}
\begin{deflist}{\texttt{xbounds}\ }
\item[\texttt{gridID}]
Grid ID, from a previous call to {\htmlref{\texttt{gridCreate}}{gridCreate}} or {\htmlref{\texttt{vlistInqVarGrid}}{vlistInqVarGrid}}.
\item[\texttt{xbounds}]
Pointer to the location into which the X-bounds are read.
                    The caller must allocate space for the returned values.

\end{deflist}
\end{minipage}

\subsubsection*{Result}

Upon successful completion {\texttt{gridInqXbounds}} returns the number of bounds and
the bounds are stored in {\texttt{xbounds}}.
Otherwise, 0 is returned and {\texttt{xbounds}} is empty.



\subsection{Define the bounds of a Y-axis: \texttt{gridDefYbounds}}
\index{gridDefYbounds}
\label{gridDefYbounds}

The function {\texttt{gridDefYbounds}} defines all bounds of the Y-axis.

\subsubsection*{Usage}

\begin{verbatim}
    void gridDefYbounds(int gridID, const double *ybounds);
\end{verbatim}

\hspace*{4mm}\begin{minipage}[]{15cm}
\begin{deflist}{\texttt{ybounds}\ }
\item[\texttt{gridID}]
Grid ID, from a previous call to {\htmlref{\texttt{gridCreate}}{gridCreate}}.
\item[\texttt{ybounds}]
Y-bounds of the grid.

\end{deflist}
\end{minipage}


\subsection{Get the bounds of a Y-axis: \texttt{gridInqYbounds}}
\index{gridInqYbounds}
\label{gridInqYbounds}

The function {\texttt{gridInqYbounds}} returns the bounds of the Y-axis.

\subsubsection*{Usage}

\begin{verbatim}
    SizeType gridInqYbounds(int gridID, double *ybounds);
\end{verbatim}

\hspace*{4mm}\begin{minipage}[]{15cm}
\begin{deflist}{\texttt{ybounds}\ }
\item[\texttt{gridID}]
Grid ID, from a previous call to {\htmlref{\texttt{gridCreate}}{gridCreate}} or {\htmlref{\texttt{vlistInqVarGrid}}{vlistInqVarGrid}}.
\item[\texttt{ybounds}]
Pointer to the location into which the Y-bounds are read.
                    The caller must allocate space for the returned values.

\end{deflist}
\end{minipage}

\subsubsection*{Result}

Upon successful completion {\texttt{gridInqYbounds}} returns the number of bounds and
the bounds are stored in {\texttt{ybounds}}.
Otherwise, 0 is returned and {\texttt{ybounds}} is empty.

