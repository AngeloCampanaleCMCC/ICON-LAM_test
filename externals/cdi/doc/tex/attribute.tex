Attributes are metadata used to describe variables or a data set.
CDI distinguishes between key attributes and user attributes.
Key attributes are described in the last chapter.

User defined attributes are additional attributes that are not interpreted by CDI.
These attributes are only available for NetCDF datasets. Here they
correspond to all attributes that are not used by CDI as key attributes.

A user defined attribute has a variable to which it is assigned, a name, a type,
a length, and a sequence of one or more values.
The attributes have to be defined after the variable is created and 
before the variables list is associated with a stream.

It is also possible to have attributes that are not associated with any variable.
These are called global attributes and are identified by using CDI\_GLOBAL as a 
variable pseudo-ID. Global attributes are usually related to the dataset as a whole.

CDI supports integer, floating point and text attributes. The data types are defined 
by the following predefined constants:

\vspace*{3mm}
\hspace*{8mm}\begin{minipage}{15cm}
\begin{deflist}{\large\texttt{CDI\_DATATYPE\_TXT \ \ }}
\item[\large\texttt{CDI\_DATATYPE\_INT16}]   16-bit integer attribute
\item[\large\texttt{CDI\_DATATYPE\_INT32}]   32-bit integer attribute
\item[\large\texttt{CDI\_DATATYPE\_FLT32}]   32-bit floating point attribute
\item[\large\texttt{CDI\_DATATYPE\_FLT64}]   64-bit floating point attribute
\item[\large\texttt{CDI\_DATATYPE\_TXT}]     Text attribute
\end{deflist}
\end{minipage}
