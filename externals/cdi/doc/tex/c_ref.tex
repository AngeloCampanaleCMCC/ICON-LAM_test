

\section*{\texttt{ 
\ifpdf
\hyperref[cdiDefAttFlt]{cdiDefAttFlt}
\else
cdiDefAttFlt
\fi
}}
\begin{verbatim}
    int cdiDefAttFlt(int cdiID, int varID, const char *name, int type, int len, 
                     const double *dp);
\end{verbatim}

Define a floating point attribute
\ifpdfoutput{}{(\ref{cdiDefAttFlt})}


\section*{\texttt{ 
\ifpdf
\hyperref[cdiDefAttInt]{cdiDefAttInt}
\else
cdiDefAttInt
\fi
}}
\begin{verbatim}
    int cdiDefAttInt(int cdiID, int varID, const char *name, int type, int len, 
                     const int *ip);
\end{verbatim}

Define an integer attribute
\ifpdfoutput{}{(\ref{cdiDefAttInt})}


\section*{\texttt{ 
\ifpdf
\hyperref[cdiDefAttTxt]{cdiDefAttTxt}
\else
cdiDefAttTxt
\fi
}}
\begin{verbatim}
    int cdiDefAttTxt(int cdiID, int varID, const char *name, int len, const char *tp);
\end{verbatim}

Define a text attribute
\ifpdfoutput{}{(\ref{cdiDefAttTxt})}


\section*{\texttt{ 
\ifpdf
\hyperref[cdiDefKeyBytes]{cdiDefKeyBytes}
\else
cdiDefKeyBytes
\fi
}}
\begin{verbatim}
    int cdiDefKeyBytes(int cdiID, int varID, int key, const unsigned char *bytes, 
                       int length);
\end{verbatim}

Define a byte array from a key
\ifpdfoutput{}{(\ref{cdiDefKeyBytes})}


\section*{\texttt{ 
\ifpdf
\hyperref[cdiDefKeyFloat]{cdiDefKeyFloat}
\else
cdiDefKeyFloat
\fi
}}
\begin{verbatim}
    int cdiDefKeyFloat(int cdiID, int varID, int key, double value);
\end{verbatim}

Define a floating point value from a key
\ifpdfoutput{}{(\ref{cdiDefKeyFloat})}


\section*{\texttt{ 
\ifpdf
\hyperref[cdiDefKeyInt]{cdiDefKeyInt}
\else
cdiDefKeyInt
\fi
}}
\begin{verbatim}
    int cdiDefKeyInt(int cdiID, int varID, int key, int value);
\end{verbatim}

Define an integer value from a key
\ifpdfoutput{}{(\ref{cdiDefKeyInt})}


\section*{\texttt{ 
\ifpdf
\hyperref[cdiDefKeyString]{cdiDefKeyString}
\else
cdiDefKeyString
\fi
}}
\begin{verbatim}
    int cdiDefKeyString(int cdiID, int varID, int key, const char *string);
\end{verbatim}

Define a string from a key
\ifpdfoutput{}{(\ref{cdiDefKeyString})}


\section*{\texttt{ 
\ifpdf
\hyperref[cdiInqAtt]{cdiInqAtt}
\else
cdiInqAtt
\fi
}}
\begin{verbatim}
    int cdiInqAtt(int cdiID, int varID, int attnum, char *name, int *typep, int *lenp);
\end{verbatim}

Get information about an attribute
\ifpdfoutput{}{(\ref{cdiInqAtt})}


\section*{\texttt{ 
\ifpdf
\hyperref[cdiInqAttFlt]{cdiInqAttFlt}
\else
cdiInqAttFlt
\fi
}}
\begin{verbatim}
    int cdiInqAttFlt(int cdiID, int varID, const char *name, int mlen, double *dp);
\end{verbatim}

Get the value(s) of a floating point attribute
\ifpdfoutput{}{(\ref{cdiInqAttFlt})}


\section*{\texttt{ 
\ifpdf
\hyperref[cdiInqAttInt]{cdiInqAttInt}
\else
cdiInqAttInt
\fi
}}
\begin{verbatim}
    int cdiInqAttInt(int cdiID, int varID, const char *name, int mlen, int *ip);
\end{verbatim}

Get the value(s) of an integer attribute
\ifpdfoutput{}{(\ref{cdiInqAttInt})}


\section*{\texttt{ 
\ifpdf
\hyperref[cdiInqAttTxt]{cdiInqAttTxt}
\else
cdiInqAttTxt
\fi
}}
\begin{verbatim}
    int cdiInqAttTxt(int cdiID, int varID, const char *name, int mlen, char *tp);
\end{verbatim}

Get the value(s) of a text attribute
\ifpdfoutput{}{(\ref{cdiInqAttTxt})}


\section*{\texttt{ 
\ifpdf
\hyperref[cdiInqKeyBytes]{cdiInqKeyBytes}
\else
cdiInqKeyBytes
\fi
}}
\begin{verbatim}
    int cdiInqKeyBytes(int cdiID, int varID, int key, unsigned char *bytes, int *length);
\end{verbatim}

Get a byte array from a key
\ifpdfoutput{}{(\ref{cdiInqKeyBytes})}


\section*{\texttt{ 
\ifpdf
\hyperref[cdiInqKeyFloat]{cdiInqKeyFloat}
\else
cdiInqKeyFloat
\fi
}}
\begin{verbatim}
    int cdiInqKeyFloat(int cdiID, int varID, int key, double *value);
\end{verbatim}

Get a floating point value from a key
\ifpdfoutput{}{(\ref{cdiInqKeyFloat})}


\section*{\texttt{ 
\ifpdf
\hyperref[cdiInqKeyInt]{cdiInqKeyInt}
\else
cdiInqKeyInt
\fi
}}
\begin{verbatim}
    int cdiInqKeyInt(int cdiID, int varID, int key, int *value);
\end{verbatim}

Get an integer value from a key
\ifpdfoutput{}{(\ref{cdiInqKeyInt})}


\section*{\texttt{ 
\ifpdf
\hyperref[cdiInqKeyString]{cdiInqKeyString}
\else
cdiInqKeyString
\fi
}}
\begin{verbatim}
    int cdiInqKeyString(int cdiID, int varID, int key, char *string, int *length);
\end{verbatim}

Get a string from a key
\ifpdfoutput{}{(\ref{cdiInqKeyString})}


\section*{\texttt{ 
\ifpdf
\hyperref[cdiInqNatts]{cdiInqNatts}
\else
cdiInqNatts
\fi
}}
\begin{verbatim}
    int cdiInqNatts(int cdiID, int varID, int *nattsp);
\end{verbatim}

Get number of attributes
\ifpdfoutput{}{(\ref{cdiInqNatts})}


\section*{\texttt{ 
\ifpdf
\hyperref[gridCreate]{gridCreate}
\else
gridCreate
\fi
}}
\begin{verbatim}
    int gridCreate(int gridtype, SizeType size);
\end{verbatim}

Create a horizontal Grid
\ifpdfoutput{}{(\ref{gridCreate})}


\section*{\texttt{ 
\ifpdf
\hyperref[gridDefNP]{gridDefNP}
\else
gridDefNP
\fi
}}
\begin{verbatim}
    void gridDefNP(int gridID, int np);
\end{verbatim}

Define the number of parallels between a pole and the equator
\ifpdfoutput{}{(\ref{gridDefNP})}


\section*{\texttt{ 
\ifpdf
\hyperref[gridDefXbounds]{gridDefXbounds}
\else
gridDefXbounds
\fi
}}
\begin{verbatim}
    void gridDefXbounds(int gridID, const double *xbounds);
\end{verbatim}

Define the bounds of a X-axis
\ifpdfoutput{}{(\ref{gridDefXbounds})}


\section*{\texttt{ 
\ifpdf
\hyperref[gridDefXsize]{gridDefXsize}
\else
gridDefXsize
\fi
}}
\begin{verbatim}
    void gridDefXsize(int gridID, SizeType xsize);
\end{verbatim}

Define the number of values of a X-axis
\ifpdfoutput{}{(\ref{gridDefXsize})}


\section*{\texttt{ 
\ifpdf
\hyperref[gridDefXvals]{gridDefXvals}
\else
gridDefXvals
\fi
}}
\begin{verbatim}
    void gridDefXvals(int gridID, const double *xvals);
\end{verbatim}

Define the values of a X-axis
\ifpdfoutput{}{(\ref{gridDefXvals})}


\section*{\texttt{ 
\ifpdf
\hyperref[gridDefYbounds]{gridDefYbounds}
\else
gridDefYbounds
\fi
}}
\begin{verbatim}
    void gridDefYbounds(int gridID, const double *ybounds);
\end{verbatim}

Define the bounds of a Y-axis
\ifpdfoutput{}{(\ref{gridDefYbounds})}


\section*{\texttt{ 
\ifpdf
\hyperref[gridDefYsize]{gridDefYsize}
\else
gridDefYsize
\fi
}}
\begin{verbatim}
    void gridDefYsize(int gridID, SizeType ysize);
\end{verbatim}

Define the number of values of a Y-axis
\ifpdfoutput{}{(\ref{gridDefYsize})}


\section*{\texttt{ 
\ifpdf
\hyperref[gridDefYvals]{gridDefYvals}
\else
gridDefYvals
\fi
}}
\begin{verbatim}
    void gridDefYvals(int gridID, const double *yvals);
\end{verbatim}

Define the values of a Y-axis
\ifpdfoutput{}{(\ref{gridDefYvals})}


\section*{\texttt{ 
\ifpdf
\hyperref[gridDestroy]{gridDestroy}
\else
gridDestroy
\fi
}}
\begin{verbatim}
    void gridDestroy(int gridID);
\end{verbatim}

Destroy a horizontal Grid
\ifpdfoutput{}{(\ref{gridDestroy})}


\section*{\texttt{ 
\ifpdf
\hyperref[gridDuplicate]{gridDuplicate}
\else
gridDuplicate
\fi
}}
\begin{verbatim}
    int gridDuplicate(int gridID);
\end{verbatim}

Duplicate a horizontal Grid
\ifpdfoutput{}{(\ref{gridDuplicate})}


\section*{\texttt{ 
\ifpdf
\hyperref[gridInqNP]{gridInqNP}
\else
gridInqNP
\fi
}}
\begin{verbatim}
    int gridInqNP(int gridID);
\end{verbatim}

Get the number of parallels between a pole and the equator
\ifpdfoutput{}{(\ref{gridInqNP})}


\section*{\texttt{ 
\ifpdf
\hyperref[gridInqSize]{gridInqSize}
\else
gridInqSize
\fi
}}
\begin{verbatim}
    SizeType gridInqSize(int gridID);
\end{verbatim}

Get the size of a Grid
\ifpdfoutput{}{(\ref{gridInqSize})}


\section*{\texttt{ 
\ifpdf
\hyperref[gridInqType]{gridInqType}
\else
gridInqType
\fi
}}
\begin{verbatim}
    int gridInqType(int gridID);
\end{verbatim}

Get the type of a Grid
\ifpdfoutput{}{(\ref{gridInqType})}


\section*{\texttt{ 
\ifpdf
\hyperref[gridInqXbounds]{gridInqXbounds}
\else
gridInqXbounds
\fi
}}
\begin{verbatim}
    SizeType gridInqXbounds(int gridID, double *xbounds);
\end{verbatim}

Get the bounds of a X-axis
\ifpdfoutput{}{(\ref{gridInqXbounds})}


\section*{\texttt{ 
\ifpdf
\hyperref[gridInqXsize]{gridInqXsize}
\else
gridInqXsize
\fi
}}
\begin{verbatim}
    SizeType gridInqXsize(int gridID);
\end{verbatim}

Get the number of values of a X-axis
\ifpdfoutput{}{(\ref{gridInqXsize})}


\section*{\texttt{ 
\ifpdf
\hyperref[gridInqXvals]{gridInqXvals}
\else
gridInqXvals
\fi
}}
\begin{verbatim}
    SizeType gridInqXvals(int gridID, double *xvals);
\end{verbatim}

Get all values of a X-axis
\ifpdfoutput{}{(\ref{gridInqXvals})}


\section*{\texttt{ 
\ifpdf
\hyperref[gridInqYbounds]{gridInqYbounds}
\else
gridInqYbounds
\fi
}}
\begin{verbatim}
    SizeType gridInqYbounds(int gridID, double *ybounds);
\end{verbatim}

Get the bounds of a Y-axis
\ifpdfoutput{}{(\ref{gridInqYbounds})}


\section*{\texttt{ 
\ifpdf
\hyperref[gridInqYsize]{gridInqYsize}
\else
gridInqYsize
\fi
}}
\begin{verbatim}
    SizeType gridInqYsize(int gridID);
\end{verbatim}

Get the number of values of a Y-axis
\ifpdfoutput{}{(\ref{gridInqYsize})}


\section*{\texttt{ 
\ifpdf
\hyperref[gridInqYvals]{gridInqYvals}
\else
gridInqYvals
\fi
}}
\begin{verbatim}
    SizeType gridInqYvals(int gridID, double *yvals);
\end{verbatim}

Get all values of a Y-axis
\ifpdfoutput{}{(\ref{gridInqYvals})}


\section*{\texttt{ 
\ifpdf
\hyperref[streamClose]{streamClose}
\else
streamClose
\fi
}}
\begin{verbatim}
    void streamClose(int streamID);
\end{verbatim}

Close an open dataset
\ifpdfoutput{}{(\ref{streamClose})}


\section*{\texttt{ 
\ifpdf
\hyperref[streamDefByteorder]{streamDefByteorder}
\else
streamDefByteorder
\fi
}}
\begin{verbatim}
    void streamDefByteorder(int streamID, int byteorder);
\end{verbatim}

Define the byte order
\ifpdfoutput{}{(\ref{streamDefByteorder})}


\section*{\texttt{ 
\ifpdf
\hyperref[streamDefRecord]{streamDefRecord}
\else
streamDefRecord
\fi
}}
\begin{verbatim}
    void streamDefRecord(int streamID, int varID, int levelID);
\end{verbatim}

Define the next record
\ifpdfoutput{}{(\ref{streamDefRecord})}


\section*{\texttt{ 
\ifpdf
\hyperref[streamDefTimestep]{streamDefTimestep}
\else
streamDefTimestep
\fi
}}
\begin{verbatim}
    int streamDefTimestep(int streamID, int tsID);
\end{verbatim}

Define a timestep
\ifpdfoutput{}{(\ref{streamDefTimestep})}


\section*{\texttt{ 
\ifpdf
\hyperref[streamDefVlist]{streamDefVlist}
\else
streamDefVlist
\fi
}}
\begin{verbatim}
    void streamDefVlist(int streamID, int vlistID);
\end{verbatim}

Define the variable list
\ifpdfoutput{}{(\ref{streamDefVlist})}


\section*{\texttt{ 
\ifpdf
\hyperref[streamInqByteorder]{streamInqByteorder}
\else
streamInqByteorder
\fi
}}
\begin{verbatim}
    int streamInqByteorder(int streamID);
\end{verbatim}

Get the byte order
\ifpdfoutput{}{(\ref{streamInqByteorder})}


\section*{\texttt{ 
\ifpdf
\hyperref[streamInqFiletype]{streamInqFiletype}
\else
streamInqFiletype
\fi
}}
\begin{verbatim}
    int streamInqFiletype(int streamID);
\end{verbatim}

Get the filetype
\ifpdfoutput{}{(\ref{streamInqFiletype})}


\section*{\texttt{ 
\ifpdf
\hyperref[streamInqTimestep]{streamInqTimestep}
\else
streamInqTimestep
\fi
}}
\begin{verbatim}
    int streamInqTimestep(int streamID, int tsID);
\end{verbatim}

Get timestep information
\ifpdfoutput{}{(\ref{streamInqTimestep})}


\section*{\texttt{ 
\ifpdf
\hyperref[streamInqVlist]{streamInqVlist}
\else
streamInqVlist
\fi
}}
\begin{verbatim}
    int streamInqVlist(int streamID);
\end{verbatim}

Get the variable list
\ifpdfoutput{}{(\ref{streamInqVlist})}


\section*{\texttt{ 
\ifpdf
\hyperref[streamOpenRead]{streamOpenRead}
\else
streamOpenRead
\fi
}}
\begin{verbatim}
    int streamOpenRead(const char *path);
\end{verbatim}

Open a dataset for reading
\ifpdfoutput{}{(\ref{streamOpenRead})}


\section*{\texttt{ 
\ifpdf
\hyperref[streamOpenWrite]{streamOpenWrite}
\else
streamOpenWrite
\fi
}}
\begin{verbatim}
    int streamOpenWrite(const char *path, int filetype);
\end{verbatim}

Create a new dataset
\ifpdfoutput{}{(\ref{streamOpenWrite})}


\section*{\texttt{ 
\ifpdf
\hyperref[streamReadVar]{streamReadVar}
\else
streamReadVar
\fi
}}
\begin{verbatim}
    void streamReadVar(int streamID, int varID, double *data, SizeType *nmiss);
\end{verbatim}

Read a variable
\ifpdfoutput{}{(\ref{streamReadVar})}


\section*{\texttt{ 
\ifpdf
\hyperref[streamReadVarF]{streamReadVarF}
\else
streamReadVarF
\fi
}}
\begin{verbatim}
    void streamReadVar(int streamID, int varID, float *data, SizeType *nmiss);
\end{verbatim}

Read a variable
\ifpdfoutput{}{(\ref{streamReadVarF})}


\section*{\texttt{ 
\ifpdf
\hyperref[streamReadVarSlice]{streamReadVarSlice}
\else
streamReadVarSlice
\fi
}}
\begin{verbatim}
    void streamReadVarSlice(int streamID, int varID, int levelID, double *data, 
                            SizeType *nmiss);
\end{verbatim}

Read a horizontal slice of a variable
\ifpdfoutput{}{(\ref{streamReadVarSlice})}


\section*{\texttt{ 
\ifpdf
\hyperref[streamReadVarSliceF]{streamReadVarSliceF}
\else
streamReadVarSliceF
\fi
}}
\begin{verbatim}
    void streamReadVarSliceF(int streamID, int varID, int levelID, float *data, 
                             SizeType *nmiss);
\end{verbatim}

Read a horizontal slice of a variable
\ifpdfoutput{}{(\ref{streamReadVarSliceF})}


\section*{\texttt{ 
\ifpdf
\hyperref[streamWriteVar]{streamWriteVar}
\else
streamWriteVar
\fi
}}
\begin{verbatim}
    void streamWriteVar(int streamID, int varID, const double *data, SizeType nmiss);
\end{verbatim}

Write a variable
\ifpdfoutput{}{(\ref{streamWriteVar})}


\section*{\texttt{ 
\ifpdf
\hyperref[streamWriteVarF]{streamWriteVarF}
\else
streamWriteVarF
\fi
}}
\begin{verbatim}
    void streamWriteVarF(int streamID, int varID, const float *data, SizeType nmiss);
\end{verbatim}

Write a variable
\ifpdfoutput{}{(\ref{streamWriteVarF})}


\section*{\texttt{ 
\ifpdf
\hyperref[streamWriteVarSlice]{streamWriteVarSlice}
\else
streamWriteVarSlice
\fi
}}
\begin{verbatim}
    void streamWriteVarSlice(int streamID, int varID, int levelID, const double *data, 
                             SizeType nmiss);
\end{verbatim}

Write a horizontal slice of a variable
\ifpdfoutput{}{(\ref{streamWriteVarSlice})}


\section*{\texttt{ 
\ifpdf
\hyperref[streamWriteVarSliceF]{streamWriteVarSliceF}
\else
streamWriteVarSliceF
\fi
}}
\begin{verbatim}
    void streamWriteVarSliceF(int streamID, int varID, int levelID, const float *data, 
                              SizeType nmiss);
\end{verbatim}

Write a horizontal slice of a variable
\ifpdfoutput{}{(\ref{streamWriteVarSliceF})}


\section*{\texttt{ 
\ifpdf
\hyperref[taxisCreate]{taxisCreate}
\else
taxisCreate
\fi
}}
\begin{verbatim}
    int taxisCreate(int taxistype);
\end{verbatim}

Create a Time axis
\ifpdfoutput{}{(\ref{taxisCreate})}


\section*{\texttt{ 
\ifpdf
\hyperref[taxisDefCalendar]{taxisDefCalendar}
\else
taxisDefCalendar
\fi
}}
\begin{verbatim}
    void taxisDefCalendar(int taxisID, int calendar);
\end{verbatim}

Define the calendar
\ifpdfoutput{}{(\ref{taxisDefCalendar})}


\section*{\texttt{ 
\ifpdf
\hyperref[taxisDefRdate]{taxisDefRdate}
\else
taxisDefRdate
\fi
}}
\begin{verbatim}
    void taxisDefRdate(int taxisID, int rdate);
\end{verbatim}

Define the reference date
\ifpdfoutput{}{(\ref{taxisDefRdate})}


\section*{\texttt{ 
\ifpdf
\hyperref[taxisDefRtime]{taxisDefRtime}
\else
taxisDefRtime
\fi
}}
\begin{verbatim}
    void taxisDefRtime(int taxisID, int rtime);
\end{verbatim}

Define the reference time
\ifpdfoutput{}{(\ref{taxisDefRtime})}


\section*{\texttt{ 
\ifpdf
\hyperref[taxisDefVdate]{taxisDefVdate}
\else
taxisDefVdate
\fi
}}
\begin{verbatim}
    void taxisDefVdate(int taxisID, int vdate);
\end{verbatim}

Define the verification date
\ifpdfoutput{}{(\ref{taxisDefVdate})}


\section*{\texttt{ 
\ifpdf
\hyperref[taxisDefVtime]{taxisDefVtime}
\else
taxisDefVtime
\fi
}}
\begin{verbatim}
    void taxisDefVtime(int taxisID, int vtime);
\end{verbatim}

Define the verification time
\ifpdfoutput{}{(\ref{taxisDefVtime})}


\section*{\texttt{ 
\ifpdf
\hyperref[taxisDestroy]{taxisDestroy}
\else
taxisDestroy
\fi
}}
\begin{verbatim}
    void taxisDestroy(int taxisID);
\end{verbatim}

Destroy a Time axis
\ifpdfoutput{}{(\ref{taxisDestroy})}


\section*{\texttt{ 
\ifpdf
\hyperref[taxisInqCalendar]{taxisInqCalendar}
\else
taxisInqCalendar
\fi
}}
\begin{verbatim}
    int taxisInqCalendar(int taxisID);
\end{verbatim}

Get the calendar
\ifpdfoutput{}{(\ref{taxisInqCalendar})}


\section*{\texttt{ 
\ifpdf
\hyperref[taxisInqRdate]{taxisInqRdate}
\else
taxisInqRdate
\fi
}}
\begin{verbatim}
    int taxisInqRdate(int taxisID);
\end{verbatim}

Get the reference date
\ifpdfoutput{}{(\ref{taxisInqRdate})}


\section*{\texttt{ 
\ifpdf
\hyperref[taxisInqRtime]{taxisInqRtime}
\else
taxisInqRtime
\fi
}}
\begin{verbatim}
    int taxisInqRtime(int taxisID);
\end{verbatim}

Get the reference time
\ifpdfoutput{}{(\ref{taxisInqRtime})}


\section*{\texttt{ 
\ifpdf
\hyperref[taxisInqVdate]{taxisInqVdate}
\else
taxisInqVdate
\fi
}}
\begin{verbatim}
    int taxisInqVdate(int taxisID);
\end{verbatim}

Get the verification date
\ifpdfoutput{}{(\ref{taxisInqVdate})}


\section*{\texttt{ 
\ifpdf
\hyperref[taxisInqVtime]{taxisInqVtime}
\else
taxisInqVtime
\fi
}}
\begin{verbatim}
    int taxisInqVtime(int taxisID);
\end{verbatim}

Get the verification time
\ifpdfoutput{}{(\ref{taxisInqVtime})}


\section*{\texttt{ 
\ifpdf
\hyperref[vlistCat]{vlistCat}
\else
vlistCat
\fi
}}
\begin{verbatim}
    void vlistCat(int vlistID2, int vlistID1);
\end{verbatim}

Concatenate two variable lists
\ifpdfoutput{}{(\ref{vlistCat})}


\section*{\texttt{ 
\ifpdf
\hyperref[vlistCopy]{vlistCopy}
\else
vlistCopy
\fi
}}
\begin{verbatim}
    void vlistCopy(int vlistID2, int vlistID1);
\end{verbatim}

Copy a variable list
\ifpdfoutput{}{(\ref{vlistCopy})}


\section*{\texttt{ 
\ifpdf
\hyperref[vlistCopyFlag]{vlistCopyFlag}
\else
vlistCopyFlag
\fi
}}
\begin{verbatim}
    void vlistCopyFlag(int vlistID2, int vlistID1);
\end{verbatim}

Copy some entries of a variable list
\ifpdfoutput{}{(\ref{vlistCopyFlag})}


\section*{\texttt{ 
\ifpdf
\hyperref[vlistCreate]{vlistCreate}
\else
vlistCreate
\fi
}}
\begin{verbatim}
    int vlistCreate(void);
\end{verbatim}

Create a variable list
\ifpdfoutput{}{(\ref{vlistCreate})}


\section*{\texttt{ 
\ifpdf
\hyperref[vlistDefTaxis]{vlistDefTaxis}
\else
vlistDefTaxis
\fi
}}
\begin{verbatim}
    void vlistDefTaxis(int vlistID, int taxisID);
\end{verbatim}

Define the time axis
\ifpdfoutput{}{(\ref{vlistDefTaxis})}


\section*{\texttt{ 
\ifpdf
\hyperref[vlistDefVar]{vlistDefVar}
\else
vlistDefVar
\fi
}}
\begin{verbatim}
    int vlistDefVar(int vlistID, int gridID, int zaxisID, int timetype);
\end{verbatim}

Define a Variable
\ifpdfoutput{}{(\ref{vlistDefVar})}


\section*{\texttt{ 
\ifpdf
\hyperref[vlistDefVarCode]{vlistDefVarCode}
\else
vlistDefVarCode
\fi
}}
\begin{verbatim}
    void vlistDefVarCode(int vlistID, int varID, int code);
\end{verbatim}

Define the code number of a Variable
\ifpdfoutput{}{(\ref{vlistDefVarCode})}


\section*{\texttt{ 
\ifpdf
\hyperref[vlistDefVarDatatype]{vlistDefVarDatatype}
\else
vlistDefVarDatatype
\fi
}}
\begin{verbatim}
    void vlistDefVarDatatype(int vlistID, int varID, int datatype);
\end{verbatim}

Define the data type of a Variable
\ifpdfoutput{}{(\ref{vlistDefVarDatatype})}


\section*{\texttt{ 
\ifpdf
\hyperref[vlistDefVarMissval]{vlistDefVarMissval}
\else
vlistDefVarMissval
\fi
}}
\begin{verbatim}
    void vlistDefVarMissval(int vlistID, int varID, double missval);
\end{verbatim}

Define the missing value of a Variable
\ifpdfoutput{}{(\ref{vlistDefVarMissval})}


\section*{\texttt{ 
\ifpdf
\hyperref[vlistDestroy]{vlistDestroy}
\else
vlistDestroy
\fi
}}
\begin{verbatim}
    void vlistDestroy(int vlistID);
\end{verbatim}

Destroy a variable list
\ifpdfoutput{}{(\ref{vlistDestroy})}


\section*{\texttt{ 
\ifpdf
\hyperref[vlistDuplicate]{vlistDuplicate}
\else
vlistDuplicate
\fi
}}
\begin{verbatim}
    int vlistDuplicate(int vlistID);
\end{verbatim}

Duplicate a variable list
\ifpdfoutput{}{(\ref{vlistDuplicate})}


\section*{\texttt{ 
\ifpdf
\hyperref[vlistInqTaxis]{vlistInqTaxis}
\else
vlistInqTaxis
\fi
}}
\begin{verbatim}
    int vlistInqTaxis(int vlistID);
\end{verbatim}

Get the time axis
\ifpdfoutput{}{(\ref{vlistInqTaxis})}


\section*{\texttt{ 
\ifpdf
\hyperref[vlistInqVarCode]{vlistInqVarCode}
\else
vlistInqVarCode
\fi
}}
\begin{verbatim}
    int vlistInqVarCode(int vlistID, int varID);
\end{verbatim}

Get the Code number of a Variable
\ifpdfoutput{}{(\ref{vlistInqVarCode})}


\section*{\texttt{ 
\ifpdf
\hyperref[vlistInqVarDatatype]{vlistInqVarDatatype}
\else
vlistInqVarDatatype
\fi
}}
\begin{verbatim}
    int vlistInqVarDatatype(int vlistID, int varID);
\end{verbatim}

Get the data type of a Variable
\ifpdfoutput{}{(\ref{vlistInqVarDatatype})}


\section*{\texttt{ 
\ifpdf
\hyperref[vlistInqVarGrid]{vlistInqVarGrid}
\else
vlistInqVarGrid
\fi
}}
\begin{verbatim}
    int vlistInqVarGrid(int vlistID, int varID);
\end{verbatim}

Get the Grid ID of a Variable
\ifpdfoutput{}{(\ref{vlistInqVarGrid})}


\section*{\texttt{ 
\ifpdf
\hyperref[vlistInqVarMissval]{vlistInqVarMissval}
\else
vlistInqVarMissval
\fi
}}
\begin{verbatim}
    double vlistInqVarMissval(int vlistID, int varID);
\end{verbatim}

Get the missing value of a Variable
\ifpdfoutput{}{(\ref{vlistInqVarMissval})}


\section*{\texttt{ 
\ifpdf
\hyperref[vlistInqVarTsteptype]{vlistInqVarTsteptype}
\else
vlistInqVarTsteptype
\fi
}}
\begin{verbatim}
    int vlistInqVarTsteptype(int vlistID, int varID);
\end{verbatim}

Get the timestep type of a Variable
\ifpdfoutput{}{(\ref{vlistInqVarTsteptype})}


\section*{\texttt{ 
\ifpdf
\hyperref[vlistInqVarZaxis]{vlistInqVarZaxis}
\else
vlistInqVarZaxis
\fi
}}
\begin{verbatim}
    int vlistInqVarZaxis(int vlistID, int varID);
\end{verbatim}

Get the Zaxis ID of a Variable
\ifpdfoutput{}{(\ref{vlistInqVarZaxis})}


\section*{\texttt{ 
\ifpdf
\hyperref[vlistNgrids]{vlistNgrids}
\else
vlistNgrids
\fi
}}
\begin{verbatim}
    int vlistNgrids(int vlistID);
\end{verbatim}

Number of grids in a variable list
\ifpdfoutput{}{(\ref{vlistNgrids})}


\section*{\texttt{ 
\ifpdf
\hyperref[vlistNvars]{vlistNvars}
\else
vlistNvars
\fi
}}
\begin{verbatim}
    int vlistNvars(int vlistID);
\end{verbatim}

Number of variables in a variable list
\ifpdfoutput{}{(\ref{vlistNvars})}


\section*{\texttt{ 
\ifpdf
\hyperref[vlistNzaxis]{vlistNzaxis}
\else
vlistNzaxis
\fi
}}
\begin{verbatim}
    int vlistNzaxis(int vlistID);
\end{verbatim}

Number of zaxis in a variable list
\ifpdfoutput{}{(\ref{vlistNzaxis})}


\section*{\texttt{ 
\ifpdf
\hyperref[zaxisCreate]{zaxisCreate}
\else
zaxisCreate
\fi
}}
\begin{verbatim}
    int zaxisCreate(int zaxistype, int size);
\end{verbatim}

Create a vertical Z-axis
\ifpdfoutput{}{(\ref{zaxisCreate})}


\section*{\texttt{ 
\ifpdf
\hyperref[zaxisDefLevels]{zaxisDefLevels}
\else
zaxisDefLevels
\fi
}}
\begin{verbatim}
    void zaxisDefLevels(int zaxisID, const double *levels);
\end{verbatim}

Define the levels of a Z-axis
\ifpdfoutput{}{(\ref{zaxisDefLevels})}


\section*{\texttt{ 
\ifpdf
\hyperref[zaxisDestroy]{zaxisDestroy}
\else
zaxisDestroy
\fi
}}
\begin{verbatim}
    void zaxisDestroy(int zaxisID);
\end{verbatim}

Destroy a vertical Z-axis
\ifpdfoutput{}{(\ref{zaxisDestroy})}


\section*{\texttt{ 
\ifpdf
\hyperref[zaxisInqLevel]{zaxisInqLevel}
\else
zaxisInqLevel
\fi
}}
\begin{verbatim}
    double zaxisInqLevel(int zaxisID, int levelID);
\end{verbatim}

Get one level of a Z-axis
\ifpdfoutput{}{(\ref{zaxisInqLevel})}


\section*{\texttt{ 
\ifpdf
\hyperref[zaxisInqLevels]{zaxisInqLevels}
\else
zaxisInqLevels
\fi
}}
\begin{verbatim}
    void zaxisInqLevels(int zaxisID, double *levels);
\end{verbatim}

Get all levels of a Z-axis
\ifpdfoutput{}{(\ref{zaxisInqLevels})}


\section*{\texttt{ 
\ifpdf
\hyperref[zaxisInqSize]{zaxisInqSize}
\else
zaxisInqSize
\fi
}}
\begin{verbatim}
    int zaxisInqSize(int zaxisID);
\end{verbatim}

Get the size of a Z-axis
\ifpdfoutput{}{(\ref{zaxisInqSize})}


\section*{\texttt{ 
\ifpdf
\hyperref[zaxisInqType]{zaxisInqType}
\else
zaxisInqType
\fi
}}
\begin{verbatim}
    int zaxisInqType(int zaxisID);
\end{verbatim}

Get the type of a Z-axis
\ifpdfoutput{}{(\ref{zaxisInqType})}
