%Every input and output file is a collection of 2D or 3D variables
%over an unlimited number of time steps.

\section{GRIB}

GRIB \cite{GRIB} (GRIdded Binary) is a standard format designed by the
World Meteorological Organization (WMO) to support the efficient 
transmission and storage of gridded meteorological data.

A GRIB record consists of a series of header sections, followed by
a bitstream of packed data representing one horizontal grid of data
values. The header sections are intended to fully describe the data
included in the bitstream, specifying information such as the
parameter, units, and precision of the data, the grid system and
level type on which the data is provided, and the date and time
for which the data are valid.

Non-numeric descriptors are enumerated in tables, such that a 1-byte
code in a header section refers to a unique description.
The WMO provides a standard set of enumerated parameter names and
level types, but the standard also allows for the definition
of locally used parameters and geometries. Any activity
that generates and distributes GRIB records must also make
their locally defined GRIB tables available to users.

The GRIB records must be sorted by time to be able to read them correctly with {\CDI}.

{\CDI} does not support the full GRIB standard. The following
data representation and level types are implemented: \\

\begin{tabular}{|r|r|l|l|}
\hline
\rowcolor[gray]{.9}
GRIB1  & GRIB2 & & \\
\rowcolor[gray]{.9}
 grid type &  template & GRIB\_API name & description \\
     0  & 3.0 & regular\_ll & Regular longitude/latitude grid \\
     3  & -- & lambert & Lambert conformal grid \\
     4  & 3.40 & regular\_gg & Regular Gaussian longitude/latitude grid \\
     4  & 3.40 & reduced\_gg & Reduced Gaussian longitude/latitude grid \\
   10  & 3.1 & rotated\_ll & Rotated  longitude/latitude grid \\
   50  & 3.50 & sh & Spherical harmonic coefficients \\
 192  & 3.100 & -- & Icosahedral-hexagonal GME grid \\
  --   & 3.101 & -- & General unstructured grid \\
\hline
\end{tabular}


\begin{tabular}{|r|r|l|l|}
\hline
\rowcolor[gray]{.9}
GRIB1  & GRIB2 & & \\
\rowcolor[gray]{.9}
 level type &  level type & GRIB\_API name & description \\
     1  &    1 & surface                        & Surface level \\
     2  &    2 & cloudBase                    & Cloud base level \\
     3  &    3 & cloudTop                     & Level of cloud tops \\
     4  &    4 & isothermZero               & Level of 0$^{\circ}$ C isotherm \\
     8  &    8 & nominalTop                 & Norminal top of atmosphere \\
     9  &    9 & seaBottom                   & Sea bottom \\
   10  &  10 & entireAtmosphere        & Entire atmosphere \\
 100  & 100 & isobaricInhPa              & Isobaric level in hPa \\
 102  & 101 & meanSea                     & Mean sea level \\
 103  & 102 & heightAboveSea          & Altitude above mean sea level \\
 105  & 103 & heightAboveGround    & Height level above ground \\
 107  & 104 & sigma                          & Sigma level \\
 109  & 105 & hybrid                         & Hybrid level      \\
 110  & 105 & hybridLayer                 & Layer between two hybrid levels   \\
 111  & 106 & depthBelowLand          & Depth below land surface    \\
 112  & 106 & depthBelowLandLayer & Layer between two depths below land surface   \\   
 113  & 107 & theta                           & Isentropic (theta) level \\
   --  & 114 & --                               & Snow level \\
 160  & 160 & depthBelowSea            & Depth below sea level    \\
 162  & 162 & --                               & Lake or River Bottom \\
 163  & 163 & --                               & Bottom Of Sediment Layer  \\
 164  & 164 & --                               & Bottom Of Thermally Active Sediment Layer  \\
 165  & 165 & --                               & Bottom Of Sediment Layer Penetrated By \\ 
         &        &                                    & Thermal Wave  \\
 166  & 166 & --                               & Mixing Layer  \\
 210  & --   & isobaricInPa                & Isobaric level in Pa \\
\hline
\end{tabular}

\subsection{GRIB edition 1}

GRIB1 is implemented in {\CDI} as an internal library and enabled per default.
The internal GRIB1 library is called CGRIBEX. This is a lightweight
version of the ECMWF GRIBEX library. CGRIBEX is written in ANSI C with a portable Fortran interface. 
The configure option \texttt{--disable-cgribex} will disable the encoding/decoding of GRIB1 records with CGRIBEX.

\subsection{GRIB edition 2}

GRIB2 is available in {\CDI} via the ECMWF ecCodes \cite{ecCodes} library.
ecCodes is an external library and not part of {\CDI}. To use GRIB2 with
{\CDI} the ecCodes library must be installed before the configuration
of the {\CDI} library. Use the configure option \texttt{--with-eccodes} to
enable GRIB2 support. 

The ecCodes library is also used to encode/decode GRIB1 records if the support for the CGRIBEX library is disabled.
This feature is not tested regulary and the status is experimental!

A single GRIB2 message can contain multiple fields. This feature is not supported in {\CDI}!

\section{NetCDF}

NetCDF \cite{NetCDF} (Network Common Data Form) is an interface for array-oriented data
access and a library that provides an implementation of the interface.
The NetCDF library also defines a machine-independent format for 
representing scientific data. Together, the interface, library, and 
format support the creation, access, and sharing of scientific data.

{\CDI} only supports the classic data model of NetCDF and arrays up to 4 dimensions.
These dimensions should only be used by the horizontal and vertical grid and the time.
The NetCDF attributes should follow the
\href{http://ftp.unidata.ucar.edu/software/netcdf/docs/conventions.html}
     {GDT, COARDS or CF Conventions}.

NetCDF is an external library and not part of {\CDI}. To use NetCDF with
{\CDI} the NetCDF library must be installed before the configuration
of the {\CDI} library. Use the configure option \texttt{--with-netcdf} to
enable NetCDF support (see \htmlref{Build}{build}).

%\subsection{ncdap}


\section{SERVICE}

SERVICE is the binary exchange format of the atmospheric general circulation model ECHAM \cite{ECHAM}.
It has a header section with 8 integer values followed by the data section.
The header and the data section have the standard Fortran blocking for binary data records.
A SERVICE record can have an accuracy of 4 or 8 bytes and the byteorder can be little or big endian.
In {\CDI} the accuracy of the header and data section must be the same.
The following Fortran code example can be used to read a SERVICE record with an accuracy of 4 bytes:

\begin{lstlisting}[language=Fortran, backgroundcolor=\color{pyellow}, basicstyle=\small, columns=flexible]
   INTEGER*4 icode,ilevel,idate,itime,nlon,nlat,idispo1,idispo2
   REAL*4 field(mlon,mlat)
      ...
   READ(unit) icode,ilevel,idate,itime,nlon,nlat,idispo1,idispo2
   READ(unit) ((field(ilon,ilat), ilon=1,nlon), ilat=1,nlat)
\end{lstlisting}

The constants \texttt{mlon} and \texttt{mlat} must be greater or equal than \texttt{nlon} and \texttt{nlat}.
The meaning of the variables are:

\vspace*{3mm}
\hspace*{8mm}\begin{minipage}{10cm}
\begin{deflist}{\texttt{idispo2 \ \ }}
\item[\texttt{icode}]    The code number
\item[\texttt{ilevel}]   The level
\item[\texttt{idate}]    The date as YYYYMMDD
\item[\texttt{itime}]    The time as hhmmss
\item[\texttt{nlon}]     The number of longitudes
\item[\texttt{nlat}]     The number of latitides
\item[\texttt{idispo1}]  For the users disposal (Not used in {\CDI})
\item[\texttt{idispo2}]  For the users disposal (Not used in {\CDI})
\end{deflist}
\end{minipage}
\vspace*{3mm}

SERVICE is implemented in {\CDI} as an internal library and enabled per default.
The configure option \texttt{--disable-service} will disable the support for the SERVICE format.

\section{EXTRA}

EXTRA is the standard binary output format of the ocean model MPIOM \cite{MPIOM}.
It has a header section with 4 integer values followed by the data section. 
The header and the data section have the standard Fortran blocking for binary data records.
An EXTRA record can have an accuracy of 4 or 8 bytes and the byteorder can be little or big endian.
In {\CDI} the accuracy of the header and data section must be the same.
The following Fortran code example can be used to read an EXTRA record with an accuracy of 4 bytes:

\begin{lstlisting}[language=Fortran, backgroundcolor=\color{pyellow}, basicstyle=\small, columns=flexible]
   INTEGER*4 idate,icode,ilevel,nsize
   REAL*4 field(msize)
      ...
   READ(unit) idate,icode,ilevel,nsize
   READ(unit) (field(isize),isize=1,nsize)
\end{lstlisting}

The constant \texttt{msize} must be greater or equal than \texttt{nsize}.
The meaning of the variables are:

\vspace*{3mm}
\hspace*{8mm}\begin{minipage}{10cm}
\begin{deflist}{\texttt{idispo2 \ \ }}
\item[\texttt{idate}]    The date as YYYYMMDD
\item[\texttt{icode}]    The code number
\item[\texttt{ilevel}]   The level
\item[\texttt{nsize}]    The size of the field
\end{deflist}
\end{minipage}
\vspace*{3mm}

EXTRA is implemented in {\CDI} as an internal library and enabled per default.
The configure option \texttt{--disable-extra} will disable the support for the EXTRA format.


\section{IEG}

IEG is the standard binary output format of the regional model REMO \cite{REMO}.
It is simple an unpacked GRIB edition 1 format. The product and grid
description sections are coded with 4 byte integer values and the
data section can have 4 or 8 byte IEEE floating point values.
The header and the data section have the standard Fortran blocking
for binary data records. The IEG format has a fixed size of 100 for the
vertical coordinate table. That means it is not possible to store more
than 50 model levels with this format.
{\CDI} supports only data on Gaussian and LonLat grids for the IEG format.

IEG is implemented in {\CDI} as an internal library and enabled per default.
The configure option \texttt{--disable-ieg} will disable the support for the IEG format.
